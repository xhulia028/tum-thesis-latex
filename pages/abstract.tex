\chapter{\abstractname}

Simulating particle interactions is a fundamental challenge in computational physics, molecular dynamics, and engineering applications. Efficiently managing these simulations is particularly important in large-scale systems, where millions of particles interact over extended periods of time. A key aspect of these simulations is the identification of particles which are eligible for pairwise force calculations. In algorithms like Verlet Lists, these particles are stored in neighbor lists, which are periodically rebuilt to maintain accuracy.

This thesis investigates the Fast-Particle-Buffer mechanism as an alternative approach to optimize the neighbor list rebuild process in AutoPas, a performance-optimized C++ library for short-range particle simulations. Instead of triggering a full rebuild whenever fast-moving particles exceed a predefined threshold, these particles are temporarily stored in a buffer, allowing the neighbor lists to be used longer before rebuilding. 

Through various test scenarios, this approach is evaluated in terms of computational efficiency and its impact on overall simulation performance. The findings suggest that while the Fast-Particle-Buffer can reduce neighbor list rebuild costs, its effectiveness depends on the simulation's characteristics. 

