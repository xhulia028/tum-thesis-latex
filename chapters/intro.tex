\chapter{Introduction}\label{chapter:introduction}


% Simulating particle interactions is a fundamental problem in computational physics, molecular dynamics, and engineering applications. Efficiently handling these simulations is crucial, particularly in large-scale systems where millions of particles interact over extended time periods. These simulations can be very computationally demanding, making runtime optimization a critical objective to improve performance and scalability. \parencite{aktulga2012parallel}

Simulating particle interactions plays a crucial role in computational physics, molecular dynamics, and engineering applications. In large-scale systems, where millions of particles interact over extended time periods, these simulations require significant computational resources. As a result, optimizing runtime is essential to improve efficiency, performance, and scalability. \parencite{aktulga2012parallel}


AutoPas is an auto-tuning C++ library designed to optimize short-range particle simulations by selecting the most efficient algorithm for a given simulation. The library implements various algorithms. However, since each simulation is unique, there is no single best algorithm that performs optimally in all cases. To address this, AutoPas uses auto-tuning, which dynamically selects the most efficient combination of internal algorithms during run-time \parencite{seckler2021autopas}.

One of the internal algorithms provided by AutoPas is Verlet Lists. A key challenge in Verlet Lists is the computational overhead caused by frequent neighbor list rebuilds. As particles move dynamically within the simulation domain, neighbor lists—which store information about neighboring particles within a defined search region—become outdated and must be recomputed periodically. This rebuilding is computationally expensive, impacting overall simulation performance.

This thesis investigates an alternative approach: the Fast-Particle-Buffer mechanism. Instead of rebuilding neighbor lists whenever a fast-moving particle enters the search region, the buffer temporarily stores these particles and avoids neighbor lists rebuilding. The goal is to analyze whether this strategy can reduce computational overhead and improve overall simulation performance.

% To evaluate this approach, various simulation scenarios were tested, each chosen to reflect different particle behaviors and interaction dynamics. The experiments were conducted using AutoPas, and the results provide insights into when and where the Fast-Particle-Buffer mechanism offers advantages.

The structure of this thesis is as follows: Chapter 2 provides a theoretical background on molecular dynamics and intorduces AutoPas. Chapter 3 presents related projects, while Chapter 4 details the implementation of the Fast-Particle-Buffer mechanism. Chapter 5 presents experimental results, analyzing the efficiency of the proposed approaches across various scenarios. Finally, Chapter 6 discusses potential future optimizations, followed by Chapter 7, which summarizes the key findings and conclusions of this work.
