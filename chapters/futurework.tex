
While the Fast-Particle-Buffer showed potential in larger simulations, it did not provide a significant performance improvement in smaller ones. Further research can focus on different aspects to enhance its efficiency.

One possible direction is studying how different hardware setups affect the buffer's performance. Running tests on various processors and memory architectures could help understand whether certain hardware configurations benefit more from the buffer approach.

Another improvement could involve combining the Fast-Particle-Buffer with Gall's adaptive partial neighbor list rebuild strategy \parencite{gall2023exploration}. Since both methods aim to reduce the cost of rebuilding neighbor lists, merging them could lead to better overall performance.

Additionally, an alternative approach to identifying fast particles could be explored. Instead of classifying each particle individually, a group-based approach could be tested. The current definition of fast particles is based on how much a particle moves relative to its previous position, determining whether it has potentially left or entered another particle's search region. However, this classification does not consider relative motion within a larger system of particles. If a group of particles moves as a unit, their positions relative to one another remain unchanged, meaning that no particle has actually moved into or out of another's search region. In this case, classifying them as fast particles and adding them to the buffer would be unnecessary because their relative interactions have not changed. By incorporating a group motion detection method, the buffer mechanism could avoid unnecessary classifications of fast particles, reducing remainder traversal costs. 


Testing the buffer in other larger and longer-running simulations could also be beneficial. While small simulations showed only minor improvements, a 6\% performance boost was observed in one of the longer simulations. This suggests that the effectiveness of the buffer may become more apparent in extended simulations. Running additional tests on different long-running scenarios could help determine whether if even greater performance gains can be achieved.

Finally, optimizing how buffer traversal is handled could further reduce remainder traversal time. Since remainder traversal is currently a major performance bottleneck, improving its efficiency could reveal the full advantages of the buffer approach.
